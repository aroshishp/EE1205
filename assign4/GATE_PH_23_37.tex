% \iffalse
\let\negmedspace\undefined
\let\negthickspace\undefined
\documentclass[journal,12pt,twocolumn]{IEEEtran}
\usepackage{cite}
\usepackage{amsmath,amssymb,amsfonts,amsthm}
\usepackage{algorithmic}
\usepackage{graphicx}
\usepackage{textcomp}
\usepackage{xcolor}
\usepackage{txfonts}
\usepackage{listings}
\usepackage{enumitem}
\usepackage{mathtools}
\usepackage{gensymb}
\usepackage{comment}
\usepackage[breaklinks=true]{hyperref}
\usepackage{tkz-euclide} 
\usepackage{listings}
\usepackage{gvv}                            \usepackage{tikz}
\usepackage{circuitikz}
\def\inputGnumericTable{}                                
\usepackage[latin1]{inputenc}                            
\usepackage{color}                                       
\usepackage{array}                                       
\usepackage{longtable}                                   
\usepackage{calc}                              
\usepackage{tikz}
\usepackage{multirow}                                    
\usepackage{hhline}                                      
\usepackage{ifthen}                            
\usepackage{caption}
\usepackage{lscape}
\usepackage{amsmath}
\newtheorem{theorem}{Theorem}[section]
\newtheorem{problem}{Problem}
\newtheorem{proposition}{Proposition}[section]
\newtheorem{lemma}{Lemma}[section]
\newtheorem{corollary}[theorem]{Corollary}
\newtheorem{example}{Example}[section]
\newtheorem{definition}[problem]{Definition}
\newcommand{\BEQA}{\begin{eqnarray}}
\newcommand{\EEQA}{\end{eqnarray}}
\newcommand{\define}{\stackrel{\triangle}{=}}
\theoremstyle{remark}
\newtheorem{rem}{Remark}

\begin{document}

\bibliographystyle{IEEEtran}
\vspace{3cm}

\title{NCERT Math 11.9.2 Q8}
\author{EE23BTECH11009 - AROSHISH PRADHAN$^{*}$% <-this % stops a space
}
\maketitle
\newpage
\bigskip
\textbf{Question:} An input voltage in the form of a square wave of frequency $1\, kHz$ is given to a circuit, which results in the output shown schematically below. Which one of the following options is the CORRECT representation of the circuit?

\begin{figure}[!h]
    \centering
    \includegraphics[width = 0.6\columnwidth]{figs/question.png}
    \caption{}
    \label{fig:ques_gate.ph.23.37}
\end{figure}

\begin{enumerate}[label = (\alph*)]
    \item 
    \begin{figure}[!h]
        \centering
        \resizebox{0.2\textwidth}{!}{\input{figs/optA}}
        \label{optA_gate.ph.23.37}
    \end{figure}

    \item 
    \begin{figure}[!h]
        \centering
        \resizebox{0.2\textwidth}{!}{\input{figs/optB}}
        \label{optA_gate.ph.23.37}
    \end{figure}

    \item 
    \begin{figure}[!h]
        \centering
        \resizebox{0.2\textwidth}{!}{\input{figs/optC}}
        \label{optA_gate.ph.23.37}
    \end{figure}

    \item 
    \begin{figure}[!h]
        \centering
        \resizebox{0.2\textwidth}{!}{\input{figs/optD}}
        \label{optA_gate.ph.23.37}
    \end{figure}
\end{enumerate}
\end{document}

