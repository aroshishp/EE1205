% \let\negmedspace\undefined
% \let\negthickspace\undefined
% \documentclass[journal,12pt,twocolumn]{IEEEtran}
% %
% \usepackage{setspace}
% % \usepackage{gensymb}
% \usepackage{xcolor}
% \usepackage{caption}
% % \usepackage{gvv}

% \singlespacing

% \usepackage[cmex10]{amsmath}
% \usepackage{mathtools}

% \usepackage{hyperref}
% \usepackage{amsthm}
% \usepackage{mathrsfs}
% \usepackage{txfonts}
% \usepackage{stfloats}
% \usepackage{cite}
% \usepackage{cases}
% \usepackage{subfig}

% \usepackage{longtable}
% \usepackage{multirow}

% \usepackage{enumitem}
% \usepackage{mathtools}

% \usepackage{listings}

% \DeclareMathOperator*{\Res}{Res}

% \renewcommand\thesection{\arabic{section}}
% \renewcommand\thesubsection{\thesection.\arabic{subsection}}
% \renewcommand\thesubsubsection{\thesubsection.\arabic{subsubsection}}

% \renewcommand\thesectiondis{\arabic{section}}
% \renewcommand\thesubsectiondis{\thesectiondis.\arabic{subsection}}
% \renewcommand\thesubsubsectiondis{\thesubsectiondis.\arabic{subsubsection}}

% \hyphenation{op-tical net-works semi-conduc-tor}

% \lstset{
% language=Python,
% frame=single, 
% breaklines=true,
% columns=fullflexible
% }
\let\negmedspace\undefined
\let\negthickspace\undefined
\documentclass[journal,12pt,twocolumn]{IEEEtran}
\usepackage{float}
\usepackage{circuitikz}
\usepackage{cite}
\usepackage{amsmath,amssymb,amsfonts,amsthm}
\usepackage{algorithmic}
\usepackage{graphicx}
\usepackage{textcomp}
\usepackage{xcolor}
\usepackage{txfonts}
\usepackage{listings}
\usepackage{amsmath}
\usepackage{enumitem}
\usepackage{mathtools}
\usepackage{gensymb}
\usepackage{comment}
\usepackage[breaklinks=true]{hyperref}
\usepackage{tkz-euclide} 
\usepackage{listings}
\usepackage{gvv}                                        
\def\inputGnumericTable{}                                 
\usepackage[latin1]{inputenc}                                
\usepackage{color}                                            
\usepackage{array}                                            
\usepackage{longtable}                                       
\usepackage{calc}  
\usepackage{caption}
\usepackage{multirow}                                         
\usepackage{hhline}                                           
\usepackage{ifthen}                                           
\usepackage{lscape}
\newtheorem{theorem}{Theorem}[section]
\newtheorem{problem}{Problem}
\newtheorem{proposition}{Proposition}[section]
\newtheorem{lemma}{Lemma}[section]
\newtheorem{corollary}[theorem]{Corollary}
\newtheorem{example}{Example}[section]
\newtheorem{definition}[problem]{Definition}
\newcommand{\BEQA}{\begin{eqnarray}}
\newcommand{\EEQA}{\end{eqnarray}}
\newcommand{\define}{\stackrel{\triangle}{=}}
\theoremstyle{remark}
\newtheorem{rem}{Remark}
\renewcommand\thesection{\arabic{section}}
\renewcommand\thesubsection{\thesection.\arabic{subsection}}
\renewcommand\thesubsubsection{\thesubsection.\arabic{subsubsection}}

\renewcommand\thesectiondis{\arabic{section}}
\renewcommand\thesubsectiondis{\thesectiondis.\arabic{subsection}}
\renewcommand\thesubsubsectiondis{\thesubsectiondis.\arabic{subsubsection}}
\lstset{
language=Python,
frame=single, 
breaklines=true,
columns=fullflexible
}
\numberwithin{equation}{subsection}
\renewcommand{\thesubsection}{\thesection.\arabic{subsection}}


\begin{document}
%

% \theoremstyle{definition}
% \newtheorem{theorem}{Theorem}[section]
% \newtheorem{problem}{Problem}
% \newtheorem{proposition}{Proposition}[section]
% \newtheorem{lemma}{Lemma}[section]
% \newtheorem{corollary}[theorem]{Corollary}
% \newtheorem{example}{Example}[section]
% \newtheorem{definition}{Definition}[section]
% %\newtheorem{algorithm}{Algorithm}[section]
% %\newtheorem{cor}{Corollary}
% \newcommand{\BEQA}{\begin{eqnarray}}
% \newcommand{\EEQA}{\end{eqnarray}}
% \newcommand{\define}{\stackrel{\triangle}{=}}

\bibliographystyle{IEEEtran}
%\bibliographystyle{ieeetr}

% \providecommand{\nCr}[2]{\,^{#1}C_{#2}} % nCr
% \providecommand{\nPr}[2]{\,^{#1}P_{#2}} % nPr
% \providecommand{\mbf}{\mathbf}
% \providecommand{\pr}[1]{\ensuremath{\Pr\left#1\right)}}
% \providecommand{\qfunc}[1]{\ensuremath{Q\left(#1\right)}}
% \providecommand{\sbrak}[1]{\ensuremath{{}\left[#1\right]}}
% \providecommand{\lsbrak}[1]{\ensuremath{{}\left[#1\right.}}
% \providecommand{\rsbrak}[1]{\ensuremath{{}\left.#1\right]}}
% \providecommand{\brak}[1]{\ensuremath{\left(#1\right)}}
% \providecommand{\lbrak}[1]{\ensuremath{\left(#1\right.}}
% \providecommand{\rbrak}[1]{\ensuremath{\left.#1\right)}}
% \providecommand{\cbrak}[1]{\ensuremath{\left\{#1\right\}}}
% \providecommand{\lcbrak}[1]{\ensuremath{\left\{#1\right.}}
% \providecommand{\rcbrak}[1]{\ensuremath{\left.#1\right\}}}
% \theoremstyle{remark}
% \newtheorem{rem}{Remark}
% \newcommand{\sgn}{\mathop{\mathrm{sgn}}}
% \providecommand{\abs}[1]{\ensuremath{\left\vert#1\right\vert}}
% \providecommand{\res}[1]{\Res\displaylimits_{#1}} 
% \providecommand{\norm}[1]{\lVert#1\rVert}
% \providecommand{\mtx}[1]{\mathbf{#1}}
% \providecommand{\mean}[1]{\ensuremath{\left[ #1 \right]}}
% \providecommand{\fourier}{\overset{\mathcal{F}}{ \rightleftharpoons}}
% \providecommand{\ztrans}{\overset{\mathcal{Z}}{ \rightleftharpoons}}

% \providecommand{\hilbert}{\overset{\mathcal{H}}{ \rightleftharpoons}}
% \providecommand{\system}{\overset{\mathcal{Z}}{ \longleftrightarrow}}
% 	% \newcommand{\solution}[2]{\textbf{Solution:}{#1}}
% \newcommand{\solution}{\noindent \textbf{Solution: }}
% \providecommand{\dec}[2]{\ensuremath{\overset{#1}{\underset{#2}{\gtrless}}}}
\numberwithin{equation}{section}


% \makeatletter
% \@addtoreset{figure}{problem}
% \makeatother

% \let\StandardTheFigure\thefigure
\renewcommand{\thefigure}{\theproblem.\arabic{figure}}
\renewcommand{\thefigure}{\theproblem}


%\numberwithin{figure}{subsection}

% \def\putbox#1#2#3{\makebox[0in][l]{\makebox[#1][l]{}\raisebox{\baselineskip}[0in][0in]{\raisebox{#2}[0in][0in]{#3}}}}
%      \def\rightbox#1{\makebox[0in][r]{#1}}
%      \def\centbox#1{\makebox[0in]{#1}}
%      \def\topbox#1{\raisebox{-\baselineskip}[0in][0in]{#1}}
%      \def\midbox#1{\raisebox{-0.5\baselineskip}[0in][0in]{#1}}

\vspace{3cm}

\title{Audio Filtering}

\author{EE23BTECH11009 - AROSHISH PRADHAN$^{*}$}

\maketitle

\tableofcontents

\renewcommand{\thefigure}{\theenumi}
\renewcommand{\thetable}{\theenumi}
\bigskip

\begin{abstract}
This manual attempts digital signal processing of an audio file.
\end{abstract}

\section{Software Installation}
Run the following commands
\begin{lstlisting}
sudo apt-get update
sudo apt-get install libffi-dev libsndfile1 python3-scipy  python3-numpy python3-matplotlib 
sudo pip install cffi pysoundfile 
\end{lstlisting}
\section{Digital Filter}
\begin{enumerate}[label=\thesection.\arabic*
,ref=\thesection.\theenumi]
\item
\label{prob:input}
Download the sound file from  
\begin{lstlisting}
wget https://raw.githubusercontent.com/gadepall/ 
EE1310/master/filter/codes/Sound_Noise.wav
\end{lstlisting}
%\href{http://tlc.iith.ac.in/img/sound/Sound_Noise.wav}{\url{http://tlc.iith.ac.in/img/sound/Sound_Noise.wav}}  
%in the link given below.
%\linebreak
\item
\label{prob:spectrogram}
You will find a spectrogram at \href{https://academo.org/demos/spectrum-analyzer}{\url{https://academo.org/demos/spectrum-analyzer}}. 

Upload the sound file that you downloaded in Problem \ref{prob:input} in the spectrogram  and play.  Observe the spectrogram. What do you find?
\\

\solution The purple areas of the spectrogram represent frequencies with low intensities (noise) while the red-yellow regions represent frequnecies with high intensities (voice). 

\begin{figure}[!h]
    \centering
    \includegraphics[width = \columnwidth]{figs/unfiltered.png}
    \caption{Spectrogram before filtering}
    \label{fig:2.2}
\end{figure}

\item
\label{prob:output}
Write the python code for removal of out of band noise and execute the code.
\\

\solution Noise in the audio is     filtered out using the following python code:
\begin{lstlisting}

import soundfile as sf
from scipy import signal

# Read .wav file
input_signal, fs = sf.read('2.wav')

# Order of the filter
order = 4

# Cutoff frequency 6kHz
cutoff_freq = 6000.0

# Digital frequency
Wn = 2 * cutoff_freq / fs

# b and a are numerator and denominator polynomials, respectively
b, a = signal.butter(order, Wn, 'low')

print(a)
print(b)

output_signal = signal.lfilter(b,a, input_signal)
# Write the output signal into a .wav file
sf.write('2_fil.wav', output_signal, fs)
\end{lstlisting} \label{py:filter}

\item
The output of the python script in Problem \ref{prob:output} is the audio file 2\_fil.wav. Play the file in the spectrogram in Problem \ref{prob:spectrogram}. What do you observe?
\\

\solution 
\begin{figure}[!h]
    \centering
    \includegraphics[width = \columnwidth]{figs/filtered.png}
    \caption{Spectrogram after filtering}
    \label{fig:2.3}
\end{figure}

The background noise (low intensities) is subdued in the audio.  Also,  the signal is blank for frequencies above 6 kHz.

\end{enumerate}

\section{Difference Equation}
\begin{enumerate}[label=\thesection.\arabic*,ref=\thesection.\theenumi]
\item Let
\begin{equation}
x(n) = \cbrak{\underset{\uparrow}{1},2,3,4,2,1}
\end{equation}
Sketch $x(n)$.\\

\solution The following code yields Fig. \ref{fig:x_n}.
\begin{lstlisting}
wget https://github.com/gadepall/EE1310/raw/master/filter/codes/3.1.py
\end{lstlisting}
\begin{figure}[!h]
    \centering
    \includegraphics[width = \columnwidth]{figs/3.1.png}
    \caption{Digital Filter Input $x(n)$}
    \label{fig:x_n}
\end{figure}

\item Let
\begin{multline}
\label{eq:iir_filter}
y(n) + \frac{1}{2}y(n-1) = x(n) + x(n-2), 
\\
 y(n) = 0, n < 0
\end{multline}
Sketch $y(n)$.
\\
\solution The following code yields Fig. \ref{fig:y_n}.
\begin{lstlisting}
wget https://github.com/gadepall/EE1310/raw/master/filter/codes/3.2.py
\end{lstlisting}
\begin{figure}[!h]
\begin{center}
\includegraphics[width=\columnwidth]{figs/3.2.png}
\end{center}
\caption{Digital Filter Output $y(n)$}
\label{fig:y_n}	
\end{figure}

\end{enumerate}
\section{Z-transform}
\begin{enumerate}[label=\thesection.\arabic*]
\item The $Z$-transform of $x(n)$ is defined as
\begin{equation}
\label{eq:4.1}
X(z)={\mathcal {Z}}\{x(n)\}=\sum _{n=-\infty }^{\infty }x(n)z^{-n}
\end{equation}
%
Show that
\begin{equation}
\label{eq:4.2}
{\mathcal {Z}}\{x(n-1)\} = z^{-1}X(z)
\end{equation}
and find
\begin{equation}
	{\mathcal {Z}}\{x(n-k)\} 
\end{equation}
\solution From \eqref{eq:4.1},
\begin{align}
{\mathcal {Z}}\{x(n-1)\} &=\sum _{n=-\infty }^{\infty }x(n-1)z^{-n}\\
n &\longrightarrow n+1\\ \nonumber
&=\sum _{n=-\infty }^{\infty }x(n)z^{-n-1} \\
&= z^{-1}\sum _{n=-\infty }^{\infty }x(n)z^{-n}\\
&= z^{-1}X(z)
\end{align}
resulting in \eqref{eq:4.2}. Similarly, it can be shown that
%
\begin{align}
{\mathcal {Z}}\{x(n-k)\} &=\sum _{n=-\infty }^{\infty }x(n-k)z^{-n}\\
n &\longrightarrow n+k \nonumber\\
&=\sum _{n=-\infty }^{\infty }x(n)z^{-n-k} \\
&= z^{-k}\sum _{n=-\infty }^{\infty }x(n)z^{-n}\\
&= z^{-k}X(z) \label{eq:4.11}
\end{align}

\item Find
%
\begin{equation}
H(z) = \frac{Y(z)}{X(z)}
\end{equation}
%
from  \eqref{eq:iir_filter} assuming that the $Z$-transform is a linear operation.
\\
\solution Using \eqref{eq:4.11} in \eqref{eq:iir_filter},
\begin{align}
Y(z) + \frac{1}{2}z^{-1}Y(z) &= X(z)+z^{-2}X(z)
\\
\implies \frac{Y(z)}{X(z)} &= \frac{1 + z^{-2}}{1 + \frac{1}{2}z^{-1}}
\label{eq:freq_resp}
\end{align}
%
\item Find the Z transform of 
\begin{equation}
\delta(n)
=
\begin{cases}
1 & n = 0
\\
0 & \text{otherwise}
\end{cases}
\end{equation}
and show that the $Z$-transform of
\begin{equation}
\label{eq:unit_step}
u(n)
=
\begin{cases}
1 & n \ge 0
\\
0 & \text{otherwise}
\end{cases}
\end{equation}
is
\begin{equation}
U(z) = \frac{1}{1-z^{-1}}, \quad \abs{z} > 1
\end{equation}
\solution
\begin{align}
{\mathcal{Z}}\{\delta(n)\} &= \sum_{n=-\infty}^{\infty}\delta(n)z^{-n}\\
&= \delta(0)z^{-0}\\
&= 1
\end{align}
and from \eqref{eq:unit_step},
\begin{align}
U(z) &= \sum _{n= 0}^{\infty}z^{-n}
\\
&=\frac{1}{1-z^{-1}}, \quad \abs{z} > 1
\end{align}
using the formula for the sum of an infinite geometric progression.
%
\item Show that 
\begin{equation}
\label{eq:anun}
a^nu(n) \system \frac{1}{1-az^{-1}} \quad \abs{z} > \abs{a}
\end{equation}
%
\solution
\begin{align}
    {\mathcal{Z}}\{a^nu(n)\} &= \sum_{n=-\infty}^{\infty}a^nu(n)z^{-n}\\
    &= \sum_{n=0}^{\infty}(az^{-1})^n\\
    &= \frac{1}{1 - az^{-1}} \quad \abs{z} > \abs{a}
\end{align}
using the formula for the sum of an infinite geometric progression.
\item 
Let
\begin{equation}
H\brak{e^{j \omega}} = H\brak{z = e^{j \omega}}.
\end{equation}
Plot $\abs{H\brak{e^{j \omega}}}$.  Comment.  $H(e^{j \omega})$ is
known as the {\em Discrete Time Fourier Transform} (DTFT) of $h(n)$.
\\
\solution The following code plots Fig. \ref{fig:dtft}.
\begin{lstlisting}
wget https://raw.githubusercontent.com/gadepall/EE1310/master/filter/codes/dtft.py
\end{lstlisting}
\begin{figure}[!h]
\centering
\includegraphics[width=\columnwidth]{figs/4.5.png}
\caption{Plot of $\abs{H\brak{e^{j\omega}}}$}
\label{fig:dtft}
\end{figure}

Substituting $z = e^{j \omega}$ in \eqref{eq:freq_resp},
\begin{align}
    \abs{H(e^{j\omega})} &= \abs{\frac{1 + e^{-2j\omega}}{1 + \frac{1}{2}e^{-j\omega}}}\\
    &= \frac{\sqrt{\brak{1 + \cos{2\omega}}^2 + \brak{\sin{2\omega}}^2}}{\sqrt{\brak{1 + \frac{\cos{\omega}}{2}}^2 + \brak{\frac{\sin{\omega}}{2}}^2}}\\
    &= \frac{4|\cos{\omega}|}{\sqrt{5 + 4\cos{\omega}}}
\end{align}
which has a fundamental period of $2\pi$:
\begin{align}
    \abs{H(e^{\brak{j\omega + 2\pi}})} &= \frac{4|\cos{\brak{\omega + 2\pi}}|}{\sqrt{5 + 4\cos{\brak{\omega + 2\pi}}}}\\
    &= \frac{4|\cos{\omega}|}{\sqrt{5 + 4\cos{\omega}}}\\
    &= \abs{H(e^{j\omega})}
\end{align}
This can be verified from the graph too.

The plot verifies the property of DTFT of a siganl that it is continuous and periodic.
\end{enumerate}

\section{Impulse Response}
\begin{enumerate}[label=\thesection.\arabic*]
\item \label{prob:impulse_resp}
Find an expression for $h(n)$ using $H(z)$, given that 
%in Problem \ref{eq:ztransab} and \eqref{eq:anun}, given that
\begin{equation}
\label{eq:impulse_resp}
h(n) \system H(z)
\end{equation}
and there is a one to one relationship between $h(n)$ and $H(z)$. $h(n)$ is known as the {\em impulse response} of the
system defined by \eqref{eq:iir_filter}.
\\
\solution From \eqref{eq:freq_resp},
\begin{align}
H(z) &= \frac{1}{1 + \frac{1}{2}z^{-1}} + \frac{ z^{-2}}{1 + \frac{1}{2}z^{-1}}
\\
\implies h(n) &= \brak{-\frac{1}{2}}^{n}u(n) + \brak{-\frac{1}{2}}^{n-2}u(n-2)
\end{align}
using \eqref{eq:anun} and \eqref{eq:4.11}.
\item Sketch $h(n)$. Is it bounded? Convergent? 
\\
\solution The following code plots Fig. \ref{fig:hn}.
\begin{lstlisting}
wget https://raw.githubusercontent.com/gadepall/EE1310/master/filter/codes/hn.py
\end{lstlisting}
\begin{figure}[!ht]
\centering
\includegraphics[width=\columnwidth]{figs/5.2.png}
\caption{Plot of $h(n)$}
\label{fig:hn}
\end{figure}
From graph, it is visible that $h(n)$ is bounded.

To check for convergence we can use the ratio test:
\begin{align}
    \lim_{n \to \infty}\abs{\frac{h(n + 1)}{h(n)}} &= \abs{\frac{\brak{-\frac{1}{2}}^{n+1} + \brak{-\frac{1}{2}}^{n-1}}{\brak{-\frac{1}{2}}^{n} + \brak{-\frac{1}{2}}^{n-2}}}\\
    &= \frac{1}{2} < 1 \label{eq:5.5}
\end{align}
Hence, $h(n)$ is convergent.

\item The system with $h(n)$ is defined to be stable if
\begin{equation}
\sum_{n=-\infty}^{\infty}h(n) < \infty \label{eq:5.6}
\end{equation}
Is the system defined by \eqref{eq:iir_filter} stable for the impulse response in \eqref{eq:impulse_resp}?

\solution Sum of infinite terms of a convergent series is finite. From \eqref{eq:5.5}, we proved that $h(n)$ was convergent therefore
\begin{equation}
\sum_{n=-\infty}^{\infty}h(n) < \infty
\end{equation}
Hence, the system with the impulse response $h(n)$ is a stable system.

\item 
Compute and sketch $h(n)$ using 
\begin{equation}
\label{eq:iir_filter_h}
h(n) + \frac{1}{2}h(n-1) = \delta(n) + \delta(n-2), 
\end{equation}
%
This is the definition of $h(n)$.
\\
\solution The following code plots Fig. \ref{fig:hndef}. Note that this is the same as Fig. 
\ref{fig:hn}. 
%
\begin{lstlisting}
wget https://raw.githubusercontent.com/gadepall/EE1310/master/filter/codes/hndef.py
\end{lstlisting}
\begin{figure}[!ht]
\centering
\includegraphics[width=\columnwidth]{figs/5.4.png}
\caption{Plot of $h(n)$ using definition}
\label{fig:hndef}
\end{figure}
%
\item Compute 
%
\begin{equation}
\label{eq:convolution}
y(n) = x(n)*h(n) = \sum_{k=-\infty}^{\infty}x(k)h(n-k) 
\end{equation} 
%
Comment. The operation in \eqref{eq:convolution} is known as
{\em convolution}.
%
\\
\solution The following code plots Fig. \ref{fig:ynconv}. Note that this is the same as 
$y(n)$ in  Fig. 
\ref{fig:y_n}. 
%
\begin{lstlisting}
wget https://raw.githubusercontent.com/gadepall/EE1310/master/filter/codes/ynconv.py
\end{lstlisting}
\begin{figure}[!ht]
\centering
\includegraphics[width=\columnwidth]{figs/5.5.png}
\caption{$y(n)$ using convolution}
\label{fig:ynconv}
\end{figure}

\item Show that
\begin{equation}
y(n) =  \sum_{k=-\infty}^{\infty}x(n-k)h(k)
\end{equation}

\solution From \eqref{eq:convolution},
\begin{align}
    y\brak{n} &= \sum_{k=-\infty}^{\infty}x\brak{k}h\brak{n - k}
\end{align}
Substitute $k \to n-k$
\begin{align}
    &= \sum_{n - k=-\infty}^{\infty}x\brak{n - k}h\brak{k} \\
    &= \sum_{k=-\infty}^{\infty}x\brak{n - k}h\brak{k}\label{eq:5.13}
\end{align}
as flipping limits does not change sum.\\

The following code plots Fig. \ref{fig:5.6}. Note that this is the same as 
$y(n)$ in  Fig. 
\ref{fig:ynconv}.
\begin{figure}[!h]
    \centering
    \includegraphics[width = \columnwidth]{figs/5.6.png}
    \caption{Plot of $y(n)$ using \eqref{eq:5.13}}
    \label{fig:5.6}
\end{figure}
\end{enumerate}

\section{DFT and FFT}
\begin{enumerate}[label=\thesection.\arabic*]
\item
Compute
\begin{equation}
X(k) \define \sum _{n=0}^{N-1}x(n) e^{-j2\pi kn/N}, \quad k = 0,1,\dots, N-1
\end{equation}
and $H(k)$ using $h(n)$.
\item Compute 
\begin{equation}
Y(k) = X(k)H(k) \label{eq:6.2}
\end{equation}
\item Compute
\begin{equation}
 y\brak{n}={\frac {1}{N}}\sum _{k=0}^{N-1}Y\brak{k}\cdot e^{j 2\pi kn/N},\quad n = 0,1,\dots, N-1
\end{equation}
\\
\solution The following code plots Fig. \ref{fig:yndft} by taking {\em Inverse Discrete Fourier Transform} (IDFT) of $Y(k)$. Note that this is also the same as 
$y(n)$ in  Fig. 
\ref{fig:y_n}. It also prints out the values of $X(k)$, $H(k)$, $Y(k)$, and $y(n)$.
%
\begin{lstlisting}
wget https://raw.githubusercontent.com/gadepall/EE1310/master/filter/codes/yndft.py
\end{lstlisting}
\begin{figure}[!ht]
\centering
\includegraphics[width=\columnwidth]{figs/6.1.png}
\caption{$y(n)$ from IDFT}
\label{fig:yndft}
\end{figure}

\item Repeat the previous exercise by computing $X(k), H(k)$ and $y(n)$ through FFT and 
IFFT.\\

\solution The code below calculates $X(k)$, $H(k)$ using FFT and plots the graph of $y(n)$ using IFFT and IDFT both (to compare).
\begin{figure}[!h]
    \centering
    \includegraphics[width = \columnwidth]{figs/6.4.png}
    \caption{Plot of $y(n)$ from IDFT and IFFT}
    \label{fig:enter-label}
\end{figure}

\newpage
\item Wherever possible, express all the above equations as matrix equations.
\\

\solution The DFT matrix is given by: 
\begin{align}
	\mtx{W} = 
	\begin{pmatrix}
		\omega^0 & \omega^0 & \ldots & \omega^0 \\
		\omega^0 & \omega^1 & \ldots & \omega^{N - 1} \\
		\vdots & \vdots & \ddots & \vdots \\
		\omega^0 & \omega^{N - 1} & \ldots & \omega^{(N -1)(N - 1)}
	\end{pmatrix}
\end{align}
where $\omega=e^{-\frac{j2\pi}{N}}$ . General DFT equation is given by:
\begin{align}
    \mtx{X} = \mtx{W}\mtx{x}
\end{align}
where
\begin{align}
	\mtx{x} = 
	\begin{pmatrix}
		x(0) \\ x(1) \\ \vdots \\ x(n - 1)
	\end{pmatrix}
\end{align}
\begin{align}
	\mtx{X} = 
	\begin{pmatrix}
		X(0) \\ X(1) \\ \vdots \\ X(n - 1)
	\end{pmatrix}
\end{align}
Then from \eqref{eq:6.2}:
\begin{align}
	\mtx{Y} = \mtx{X}\odot\mtx{H} = \brak{\mtx{W}\mtx{x}}\odot\brak{\mtx{W}\mtx{h}}
\end{align}
where $\odot$ represents the Hadamard product which multiplies corresponding elements of matrices of same size.

The below code computes $y\brak{n}$ by DFT Matrix and then plots it.
\begin{lstlisting}
https://github.com/dhanushnayakh03/EE1205/tree/main/Audio_%20Filter/codes/5.5.py
\end{lstlisting}
\begin{figure}[!h]
    \centering
    \includegraphics[width = \columnwidth]{figs/6.5.png}
    \caption{Plot of $y(n)$ using matrix method}
    \label{fig:enter-la}
\end{figure}
\end{enumerate}
\section{Exercises}

Answer the following questions by looking at the python code in Problem \ref{prob:output}.
\begin{enumerate}[label=\thesection.\arabic*]
\item
The command
\begin{lstlisting}
	output_signal = signal.lfilter(b, a, input_signal)
	\end{lstlisting}
in Problem \ref{prob:output} is executed through the following difference equation
\begin{equation}
\label{eq:iir_filter_gen}
 \sum _{m=0}^{M}a\brak{m}y\brak{n-m}=\sum _{k=0}^{N}b\brak{k}x\brak{n-k}
\end{equation}
%
where the input signal is $x(n)$ and the output signal is $y(n)$ with initial values all 0. Replace
\textbf{signal.filtfilt} with your own routine and verify.
\\

\solution The code below plots the output of scipy.signal.lfilter and the output of custom function on the same graph.
\begin{lstlisting}
https://github.com/dhanushnayakh03/EE1205/tree/main/Audio_%20Filter/codes/5.5.py
\end{lstlisting}
\newpage
\begin{figure}[!h]
    \centering
    \includegraphics[width = \columnwidth]{figs/lfilter.png}
    \caption{Output of signal.lfilter and custom routine}
    \label{fig:7.1}
\end{figure}
Both the plots overlap, indicating that the custom filter code does the same function as scipy.signal.lfilter.
\item Repeat all the exercises in the previous sections for the above $a$ and $b$.\\

\solution The code in \ref{py:filter} calculates values of a and b and prints them:
$
\begin{array}{c}
  a = [ 1.000         \quad -1.792 \quad 1.518 \quad -0.608 \quad 0.098] \\
b = [0.013\quad 0.054 \quad 0.081\quad 0.054 \quad 0.013]
\end{array}
$
Now, using \eqref{eq:iir_filter_gen}, difference equation:
\begin{multline}
a\brak{0}y\brak{n} + a\brak{1}y\brak{n-1}+a\brak{2}y\brak{n-2}\\
+a\brak{3}y\brak{n-3}+a\brak{4}y\brak{n-4} =   b\brak{0}x\brak{n} \\
+ b\brak{1}x\brak{n-1}+b\brak{2}x\brak{n-2}+b\brak{3}x\brak{n-3}\\
+b\brak{4}x\brak{n-4} 
\end{multline}
Substituting,
\begin{multline}
    y\brak{n} -1.792y\brak{n-1}+1.518y\brak{n-2}\\
-0.608y\brak{n-3}+0.098y\brak{n-4} = 0.013x\brak{n} \\
+ 0.054x\brak{n-1}+0.081x\brak{n-2}+0.054x\brak{n-3}\\
+0.013x\brak{n-4} 
\end{multline}
The rational transfer function describing this filter in the z-transform domain is:
\begin{align}
     H(z) &= \frac{b(0) + b(1) z^{-1} + b(2) z^{-2} + \ldots + b(N) z^{-N}}{a(0) + a(1) z^{-1} + a(2) z^{-2} + \ldots + a(M) z^{-M}}\\ \label{eq:7.4}
    &= \frac{\sum_{k = 0}^{N}b(k)z^{-k}}{\sum_{k = 0}^{M}a(k)z^{-k}}
\end{align}
In our case, $M=N=4$.

Now, the partial fraction of \eqref{eq:7.4} is given by:
\begin{align}
   H\brak{z}&= \sum_{i}\frac{r(i)}{1 - p(i)z^{-1}} + \sum_{j}k(j)z^{-j}\label{eq:7.6}
\end{align}
Values of $r(i)$, $p(i)$ and $k(j)$ are calculated using scipy.signal.residuez, which returns the above mentioned series:
\begin{table}[!h]
    \centering
    \renewcommand\thetable{1}
    \resizebox{0.51\textwidth}{!}{
    \begin{tabular}{|c|c|c|}
    \hline
    \textbf{$r\brak{i}$} & \textbf{$p\brak{i}$} & \textbf{$k\brak{i}$} \\ \hline
    $0.28018185-1.23886252j$ &$0.38674749+0.17013423j$&$0.13702919$  \\ \hline
    $0.28018185+1.23886252j$ &$0.38674749-0.17013423j$&$-$  \\ \hline
    $-0.3419458 +0.19576406j$ &$0.50928445+0.54087922j$&$-$  \\ \hline
    $-0.3419458 -0.19576406j$ &$0.50928445-0.54087922j$&$-$  \\ \hline
    \end{tabular}}
    \caption{Values of $r(i)$, $p(i)$, $k(i)$}
    \label{tab:residuez}
\end{table}


Inverse of \eqref{eq:7.6} is given using:
\begin{align}
    a^{n}u\brak{n} &\system \frac{1}{1-az^{-1}}\\
    \delta\brak{n-k} &\system z^{-k}\\
    \implies h(n) &= \sum_{i}r(i)[p(i)]^nu(n) + \sum_{j}k(j)\delta(n - j)
\end{align}

\textbf{Plot of h(n)}:
\begin{figure}[!h]
    \centering
    \includegraphics[width = \columnwidth]{figs/7.2hn.png}
    \caption{Plot of $h(n)$}
    \label{fig:7.2}
\end{figure}
\newpage
\begin{figure}[!h]
    \centering
    \includegraphics[width = \columnwidth]{figs/7.2zp.png}
    \caption{Pole-Zero Plot}
    \label{fig:7.2zp}
\end{figure}
There are complex poles, so $h(n)$ has a damped sinusoidal form.\\

\textbf{Stability of System}
\begin{align}
H\brak{z} &= \sum_{n = 0}^{\infty} h\brak{n}z^{-n}\\
\implies H(1)&= \sum_{n = 0}^{\infty}h(n) \\
&= \frac{\sum_{k = 0}^{N}b(k)}{\sum_{k = 0}^{M}a(k)}< \infty
\end{align}
as both $a\brak{k}$ and $b\brak{k}$ are finite length sequences.\\
Then, \eqref{eq:5.6} implies $h(n)$ is impulse response of a stable system.
\\

\newpage
\textbf{Frequency Response of Butterworth Filter}
\begin{figure}[!h]
    \centering
    \includegraphics[width = \columnwidth]{figs/7.2hw.png}
    \caption{Plot of Frequency Response}
    \label{fig:7.2hw}
\end{figure}
\\
\textbf{Frequency Response of Butterworth Filter in Analog Domain}

To convert to analog domain, we can use the Bilinear Transform where we substitute:
\begin{align}
    z=\frac{1+\frac{sT}{2}}{{1-\frac{sT}{2}}}
\end{align}
\begin{figure}[!h]
    \centering
    \includegraphics[width = \columnwidth]{figs/7.2bt.png}
    \caption{Plot of Frequency Response in Analog Domain}
    \label{fig:7.2bt}
\end{figure}

\item Implement your own FFT routine in C and call this FFT in python
\\
\solution The below C code implements the FFT algorithm:
\begin{lstlisting}
    k
\end{lstlisting}
Run the following command to generate a shared library '7.3.so':
\begin{lstlisting}
    gcc -shared -o 7.3.so -fPIC 7.3.c
\end{lstlisting}
The C code is called in the following Python code and output is printed. It can be seen that the same output is printed through both codes.
\\
\item Find the Time Complexities of computing $y(n)$ using FFT/IFFT and convolution and compare. 
\\
\solution The below codes generate and plot the time complexities of computing $y(n)$ using FFT/IFFT and Convolution:
\begin{lstlisting}
    iva
\end{lstlisting}
\begin{figure}[!h]
    \centering
    \includegraphics[width = \columnwidth]{figs/7.4.png}
    \caption{Comparison of Time Complexities}
    \label{fig:7.4}
\end{figure}
Time complexity of FFT/IFFT method is $O(n\log(n))$ and that of Convolution method is $O(n^2)$.
\\
\item What is the sampling frequency of the input signal?
\\
\solution
Sampling frequency (fs) = $44100$ Hz. It can be printed from \ref{py:filter}.
\\
\item
What is type, order and  cutoff-frequency of the above butterworth filter
\\
\solution
The given Butterworth Filter is low pass with order=4 and cutoff-frequency = 6kHz.
\\
\item
Modifying the code with different input parameters and to get the best possible output.
\\
\solution The best filtered audio output was obtained by setting order to 4 and keeping cutoff frequency at 6000 Hz. These parameters were used for the spectrogram output as well as for this section.
\end{enumerate}

\end{document}

