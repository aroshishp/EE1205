% \iffalse
\let\negmedspace\undefined
\let\negthickspace\undefined
\documentclass[journal,12pt,twocolumn]{IEEEtran}
\usepackage{cite}
\usepackage{amsmath,amssymb,amsfonts,amsthm}
\usepackage{algorithmic}
\usepackage{graphicx}
\usepackage{textcomp}
\usepackage{xcolor}
\usepackage{txfonts}
\usepackage{listings}
\usepackage{enumitem}
\usepackage{mathtools}
\usepackage{gensymb}
\usepackage{comment}
\usepackage[breaklinks=true]{hyperref}
\usepackage{tkz-euclide} 
\usepackage{listings}
\usepackage{gvv}  
\usepackage{tikz}
\usepackage{circuitikz} 
\usepackage{caption}

\def\inputGnumericTable{}                                
\usepackage[latin1]{inputenc}                 
\usepackage{color}                            
\usepackage{array}                            
\usepackage{longtable}                        
\usepackage{calc}                            
\usepackage{multirow}                      
\usepackage{hhline}                           
\usepackage{ifthen}                          
\usepackage{lscape}
\usepackage{amsmath}
\newtheorem{theorem}{Theorem}[section]
\newtheorem{problem}{Problem}
\newtheorem{proposition}{Proposition}[section]
\newtheorem{lemma}{Lemma}[section]
\newtheorem{corollary}[theorem]{Corollary}
\newtheorem{example}{Example}[section]
\newtheorem{definition}[problem]{Definition}
\newcommand{\BEQA}{\begin{eqnarray}}
\newcommand{\EEQA}{\end{eqnarray}}
\newcommand{\define}{\stackrel{\triangle}{=}}
\theoremstyle{remark}
\newtheorem{rem}{Remark}



\begin{document}

\bibliographystyle{IEEEtran}
\vspace{3cm}

\title{NCERT Physics 12.7 Q21}
\author{EE23BTECH11009 - AROSHISH PRADHAN$^{*}$% <-this % stops a space
}
\maketitle
\newpage
\bigskip
\textbf{Question:} 
Obtain the resonant frequency and Q-factor of a series LCR circuit
with $\text{L} = 3.0\, \text{H}$, $\text{C} = 27\, \mu\text{F}$, and $\text{R} = 7.4\, \Omega$. It is desired to improve the
sharpness of the resonance of the circuit by reducing its `full width at half maximum' by a factor of 2. Suggest a suitable way.\\
\\
\textbf{Solution: }
Given parameters are:

\begin{table}[h]
    \centering
    \resizebox{4 cm}{!}{
    \begin{tabular}{c|c}
    \hline
    Parameter & Value\\
    \hline
      L   &  $3.0\, \text{H}$\\
      C   &  $27\, \mu\text{F}$\\
      R   &  $7.4\, \Omega$\\
    \hline
    \end{tabular}}
    \vspace{10 pt}
    \caption{Given Parameters}
    \label{tab:my_label}
\end{table}

\noindent Resonance Frequency ($\omega_0$) is given by:

\begin{center}
$\omega_0 = \dfrac{1}{\sqrt{LC}}$ 
\end{center}
Substituting values of L and C gives:
\begin{align}
\omega_0 &= \dfrac{1}{\sqrt{3\cdot27\times10^{-6}}}\\
&= \dfrac{10^{3}}{9}\, s^{-1}\\
&= 111.12\, s^{-1} 
\end{align}
Quality Factor (Q) is given by:
\begin{center}
    $\text{Q} = \dfrac{1}{R}\cdot\sqrt{\dfrac{L}{C}}$
\end{center}
Substituting values of R, L and C gives:
\begin{align}
    \text{Q} &= \dfrac{1}{7.4}\cdot\sqrt{\dfrac{3}{27\times10^{{-6}}}}  \\  
    &= \dfrac{10^{3}}{22.2}\\
    &\approx45
\end{align}
To reduce the full width at half maximum by a factor of 2, the quality factor needs to be doubled. One way of doing this is to reduce the resistance by a factor of 2.
\begin{align}
    \text{R'} &= \dfrac{R}{2}\\
    &= \dfrac{7.4}{2}\Omega\\
    &= 3.7\Omega
\end{align}

\begin{figure}[h]
 \centering
    \begin{circuitikz}
    \draw(0, 0) -- (1, 0);
    \draw(1, 0) to [L, l = $3.0\text{H}$](2, 0);
    \draw(2, 0) -- (3, 0);
    \draw(3, 0) to [C, l = $27\, \mu\text{F}$](4, 0);
    \draw(4, 0) -- (5, 0);
    \draw(5, 0) to [R, l = $7.4\Omega$](6, 0);
    \draw(0, 0) -- (0, -2);
    \draw(6, 0) -- (7, 0);
    \draw(7, 0) -- (7, -2);
    \draw(0, -2) -- (3, -2);
    \draw(7, -2) -- (7, -2);
    \draw(3, -2) to [sV, l = $V(t)$](4, -2);
    \draw(4, -2) -- (7, -2);
    \end{circuitikz}
    \caption{LCR Circuit}
    \label{fig:enter-label}
\end{figure}

\textbf{Frequency Response of the Circuit}

This is a series LCR circuit, with the elements in series with the voltage source. Applying Kirchhoff's Voltage Law (KVL), we get:

\begin{equation}
V_R + V_L + V_C = V(t)
\end{equation}

where $V_R$, $V_L$ and $V_C$ are the voltages across R, L and C respectively and $V(t)$ is the time-varying voltage source.\\
\\
Substituting,
\begin{equation}
    V_R = R\cdot I(t)
\end{equation}
\begin{equation}
    V_L = L\dfrac{dI(t)}{dt}
\end{equation}
\begin{equation}
    V_C = V(0) + \dfrac{1}{C}\int_{0}^{t}I(\tau)\, d\tau
\end{equation}

\noindent into equation (10), we get:

\begin{equation}
    R\cdot I(t) + L\dfrac{dI(t)}{dt} + V(0) + \dfrac{1}{C}\int_{0}^{t}I(\tau)\, d\tau = V(t)
\end{equation}

The response of the circuit can be analysed at the transient and steady state by using the Laplace Transform.\\
\\
The Laplace Transform $F(s)$ of a function $f(t)$, defined for all real $t > 0$, is defined by

\begin{equation}
    \mathcal{L}\{f\} = F(s) = \int_{0}^{\infty}f(t)\mathrm{e}^{-st}dt
\end{equation}

where s is a complex frequency domain parameter, i.e. $s = \alpha + \iota\omega$ ($\alpha, \omega \in \mathbf{R}$)

Applying the Laplace Transform to equation (14), we get

\begin{equation}
    V(s) = I(s)\left(R + Ls + \dfrac{1}{sC}\right)
\end{equation}

\begin{equation}
    \Rightarrow I(s) = \dfrac{V(s)}{\left(R + Ls + \dfrac{1}{sC}\right)}
\end{equation}

The term $\dfrac{I(s)}{V(s)}$ is called the Laplace Admittance Y(s).

\begin{equation}
     \Rightarrow Y(s) = \dfrac{I(s)}{V(s)} = \dfrac{s}{L\left(s^2 + \dfrac{R}{L}s + \dfrac{1}{LC}\right)}
\end{equation}

We now define two terms: Neper Frequency ($\alpha$) and Angular Resonance Frequency ($\omega_0$).

Neper Frequency or Attenuation is a measure of how fast the transient response of a circuit will die out after the source has been removed.\\
\\
For a series LCR circuit,

\begin{equation}
    \alpha = \dfrac{R}{2L}
\end{equation}

\begin{equation}
    \omega_0 = \dfrac{1}{\sqrt{LC}}
\end{equation}
Equation (18) can then be written as

\begin{equation}
    Y(s) = \dfrac{s}{L\left(s^2 + 2\alpha s + \omega_0^2\right)}
\end{equation}
The poles of Y(s) are the values of s for which $Y(s) \rightarrow \infty$, i.e.

\begin{equation}
    s^2 + 2\alpha s + \omega_0^2 = 0
\end{equation}

\begin{equation}
    \Rightarrow s = -\alpha \pm \sqrt{\alpha^2 - \omega_0^2}
\end{equation}
which are identical to the roots of the characteristic equation of equation (14).\\
\\
For the given values of R, L and C we get
\begin{equation}
    \alpha = \dfrac{R}{2L} = 1.234\,  s^{-1}
\end{equation}

\begin{equation}
    \omega_0 = \dfrac{1}{\sqrt{LC}} = 111.12\, s^{-1}
\end{equation}

As $\omega_0>\alpha$, values of s are imaginary. The frequency response of this circuit is therefore underdamped. The solution for $I(t)$ is given by inverse Laplace transform of $I(s)$:

\begin{equation}
    I(t) = \dfrac{1}{L}\int_{0}^{t}V(t-\tau)\mathrm{e}^{-\alpha\tau}\left[\cosh(\omega_d \tau) - \dfrac{\alpha}{\omega_d}\sinh(\omega_d \tau)\right]d\tau
\end{equation}
where $\omega_d = \sqrt{\omega_0^2 - \alpha^2} = 111.11\, s^{-1}$ is the damped frequency of oscillation.\\
\\
The integral yields the transient response:
\begin{equation}
    I(t) = \mathrm{e}^{-\alpha t}\left[B_1\cos(\omega_d t) + B_2\sin(\omega_d t)\right]
\end{equation}
 
where $B_1$ and $B_2$ are constants. Substituting $\alpha = 1.234\, s^{-1}$ and $\omega_d = 111.11\, s^{-1}$ we get

\begin{equation}
    I(t) = \mathrm{e}^{-1.234t}\left[B_1\cos(111.11t) + B_2\sin(111.11t)\right]
\end{equation}
where t is in seconds.
Constants $B_1$ and $B_2$ can be determined by the initial conditions of the given circuit. 

The steady-state response of the circuit will depend on the nature of input $V(t)$. The transient and steady-state responses of the circuit can be added to give the net equation of $I(t)$.

The current $I(t)$ will fluctuate in a sinusoidal manner and decay over time. The frequency response of the circuit is underdamped response.



\renewcommand{\thefigure}{\theenumi}
\renewcommand{\thetable}{\theenumi}

\end{document}
