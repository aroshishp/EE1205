% \iffalse
\let\negmedspace\undefined
\let\negthickspace\undefined
\documentclass[journal,12pt,twocolumn]{IEEEtran}
\usepackage{cite}
\usepackage{amsmath,amssymb,amsfonts,amsthm}
\usepackage{algorithmic}
\usepackage{graphicx}
\usepackage{textcomp}
\usepackage{xcolor}
\usepackage{txfonts}
\usepackage{listings}
\usepackage{enumitem}
\usepackage{mathtools}
\usepackage{gensymb}
\usepackage{comment}
\usepackage[breaklinks=true]{hyperref}
\usepackage{tkz-euclide} 
\usepackage{listings}
\usepackage{gvv}  
\usepackage{tikz}
\usepackage{circuitikz} 
\usepackage{caption}

\def\inputGnumericTable{}                                
\usepackage[latin1]{inputenc}                 
\usepackage{color}                            
\usepackage{array}                            
\usepackage{longtable}                        
\usepackage{calc}                            
\usepackage{multirow}                      
\usepackage{hhline}                           
\usepackage{ifthen}                          
\usepackage{lscape}
\usepackage{amsmath}
\newtheorem{theorem}{Theorem}[section]
\newtheorem{problem}{Problem}
\newtheorem{proposition}{Proposition}[section]
\newtheorem{lemma}{Lemma}[section]
\newtheorem{corollary}[theorem]{Corollary}
\newtheorem{example}{Example}[section]
\newtheorem{definition}[problem]{Definition}
\newcommand{\BEQA}{\begin{eqnarray}}
\newcommand{\EEQA}{\end{eqnarray}}
\newcommand{\define}{\stackrel{\triangle}{=}}
\theoremstyle{remark}
\newtheorem{rem}{Remark}

\begin{document}

\bibliographystyle{IEEEtran}
\vspace{3cm}

\title{NCERT Physics 12.7 Q21}
\author{EE23BTECH11009 - AROSHISH PRADHAN$^{*}$% <-this % stops a space
}
\maketitle
\newpage
\bigskip
\textbf{Question:} 
Obtain the resonant frequency and Q-factor of a series LCR circuit
with $\text{L} = 3.0\, \text{H}$, $\text{C} = 27\, \mu\text{F}$, and $\text{R} = 7.4\, \Omega$. It is desired to improve the
sharpness of the resonance of the circuit by reducing its `full width at half maximum' by a factor of 2. Suggest a suitable way.\\
\\
\textbf{Solution: }
Given parameters are:

\begin{table}[h]
    \centering
    \resizebox{6 cm}{!}{
    \begin{table}[!h]
    \centering
    \begin{tabular}{|c|c|c|}
    \hline
       \textbf{Symbol}  & \textbf{Value} &  \textbf{Description}\\
    \hline
       $V_{in}$  &  &  Input Voltage\\
    \hline
        $V_{out}$ & & Output Voltage\\
    \hline
        $f$ & $1000Hz$ & Input Wave Frequency\\
    \hline
        $T$ & $\dfrac{1}{f} = 10^{-3} s$ & Input Wave Time Period\\
    \hline
        \multirow{4}{*}{$R$} & (a) $0.5k\Omega$ & \multirow{4}{*}{Resistance}\\
        \cline{2-2}
        & (b) $5k\Omega$ &\\
        \cline{2-2}
        & (c) $0.5k\Omega$ &\\
        \cline{2-2}
        & (d) $5k\Omega$ &\\
    \hline
        \multirow{4}{*}{$C$} & (a) $0.1\mu F$ & \multirow{4}{*}{Capacitance}\\
        \cline{2-2}
        & (b) $1\mu F$ &\\
        \cline{2-2}
        & (c) $0.1\mu F$ &\\
        \cline{2-2}
        & (d) $1\mu F$ &\\
    \hline
        $\tau$ & $RC$ & Time Constant\\
    \hline
    \end{tabular}
    \caption{Given Parameters}
    \label{tab:1_gate.23.ph.37}
\end{table}
}
    \vspace{6 pt}
    \caption{Given Parameters}
    \label{tab:my_label}
\end{table}

\begin{figure}[h]
 \centering
    \begin{circuitikz}
    \draw(0, 0) -- (1, 0);
    \draw(1, 0) to [L, l = $3.0\text{H}$](2, 0);
    \draw(2, 0) -- (3, 0);
    \draw(3, 0) to [C, l = $27\, \mu\text{F}$](4, 0);
    \draw(4, 0) -- (5, 0);
    \draw(5, 0) to [R, l = $7.4\Omega$](6, 0);
    \draw(0, 0) -- (0, -2);
    \draw[->] (0, -1) node[left] {$I(t)$} -- (0, -1);
    \draw(6, 0) -- (7, 0);
    \draw(7, 0) -- (7, -2);
    \draw(0, -2) -- (3, -2);
    \draw(7, -2) -- (7, -2);
    \draw(3, -2) to [sV, l = $V(t)$](4, -2);
    \draw(4, -2) -- (7, -2);
\end{circuitikz}

    \caption{LCR Circuit}
    \label{fig:enter-label}
\end{figure}

\textbf{Frequency Response of the Circuit}

This is a series LCR circuit, with the elements in series with the voltage source. Applying Kirchhoff's Voltage Law (KVL), we get:

\begin{equation}
V_R + V_L + V_C = V(t)
\end{equation}

where $V_R$, $V_L$ and $V_C$ are the voltages across R, L and C respectively and $V(t)$ is the time-varying voltage source.\\

The circuit can be analysed in the frequency domain instead of the time domain by applying the Laplace Transform. This produces the corresponding impedances of the elements, which allows to solve algebraic equations instead of differential equations.\\

Elements and their corresponding impedances are given in the table below:

\begin{table}[h]
    \centering
    \resizebox{6 cm}{!}{
    \begin{table}[!h]
    \centering
    \begin{tabular}{|c|c|}
    \hline
       $s_k$  &  \textbf{Value}\\
    \hline
       $s_0$  & $-0.162 + 1.003j$\\
    \hline
       $s_1$  & $-0.391 + 0.415j$\\
    \hline
       $s_2$  & $-0.391 - 0.415j$\\
    \hline
       $s_3$  & $-0.162 - 1.003j$\\
    \hline
    \end{tabular}
    \caption{Poles on left half of complex plane}
    \label{tab:2}
\end{table}

    }
    \vspace{6 pt}
    \caption{Impedances}
    \label{tab:my_label} 
\end{table}

where $s$ is a complex variable. The circuit can now be redrawn as:

\begin{figure}[h]
 \centering
    \begin{circuitikz}
    \draw(0, 0) -- (1, 0);
    \draw(1, 0) to [L, l = $sL$](2, 0);
    \draw(2, 0) -- (3, 0);
    \draw(3, 0) to [C, l = $\frac{1}{sC}$](4, 0);
    \draw(4, 0) -- (5, 0);
    \draw(5, 0) to [R, l = $R$](6, 0);
    \draw(0, 0) -- (0, -2);
    \draw[->] (0, -1) node[left] {$I(s)$} -- (0, -1);
    \draw(6, 0) -- (7, 0);
    \draw(7, 0) -- (7, -2);
    \draw(0, -2) -- (3, -2);
    \draw(7, -2) -- (7, -2);
    \draw(3, -2) to [sV, l = $V(s)$](4, -2);
    \draw(4, -2) -- (7, -2);
\end{circuitikz}

    \caption{LCR Circuit}
    \label{fig:enter-label}
\end{figure}

Using the impedances of R, L and C from TABLE II in equation (1), we get

\begin{equation}
    V(s) = R\cdot I(s) + sL\cdot I(s) + \dfrac{1}{sC}\cdot I(s)
\end{equation}

\begin{equation}
    \Rightarrow V(s) = I(s)\left(R + Ls + \dfrac{1}{sC}\right)
\end{equation}

\begin{equation}
    \Rightarrow I(s) = \dfrac{V(s)}{\left(R + Ls + \dfrac{1}{sC}\right)}
\end{equation}

The term $\dfrac{I(s)}{V(s)}$ is called the Laplace Admittance Y(s).

\begin{equation}
     \Rightarrow Y(s) = \dfrac{I(s)}{V(s)} = \dfrac{s}{L\left(s^2 + \dfrac{R}{L}s + \dfrac{1}{LC}\right)}
\end{equation}

\renewcommand{\thefigure}{\theenumi}
\renewcommand{\thetable}{\theenumi}

\end{document}
