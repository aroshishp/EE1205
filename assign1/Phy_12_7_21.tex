% \iffalse
\let\negmedspace\undefined
\let\negthickspace\undefined
\documentclass[journal,12pt,twocolumn]{IEEEtran}
\usepackage{cite}
\usepackage{amsmath,amssymb,amsfonts,amsthm}
\usepackage{algorithmic}
\usepackage{graphicx}
\usepackage{textcomp}
\usepackage{xcolor}
\usepackage{txfonts}
\usepackage{listings}
\usepackage{enumitem}
\usepackage{mathtools}
\usepackage{gensymb}
\usepackage{comment}
\usepackage[breaklinks=true]{hyperref}
\usepackage{tkz-euclide} 
\usepackage{listings}
\usepackage{gvv}  
\usepackage{tikz}
\usepackage{circuitikz} 
\usepackage{caption}

\def\inputGnumericTable{}                                
\usepackage[latin1]{inputenc}                 
\usepackage{color}                            
\usepackage{array}                            
\usepackage{longtable}                        
\usepackage{calc}                            
\usepackage{multirow}                      
\usepackage{hhline}                           
\usepackage{ifthen}                          
\usepackage{lscape}
\usepackage{amsmath}
\newtheorem{theorem}{Theorem}[section]
\newtheorem{problem}{Problem}
\newtheorem{proposition}{Proposition}[section]
\newtheorem{lemma}{Lemma}[section]
\newtheorem{corollary}[theorem]{Corollary}
\newtheorem{example}{Example}[section]
\newtheorem{definition}[problem]{Definition}
\newcommand{\BEQA}{\begin{eqnarray}}
\newcommand{\EEQA}{\end{eqnarray}}
\newcommand{\define}{\stackrel{\triangle}{=}}
\theoremstyle{remark}
\newtheorem{rem}{Remark}

\begin{document}

\bibliographystyle{IEEEtran}
\vspace{3cm}

\title{NCERT Physics 12.7 Q21}
\author{EE23BTECH11009 - AROSHISH PRADHAN$^{*}$% <-this % stops a space
}
\maketitle
\newpage
\bigskip
\textbf{Question:} 
Obtain the resonant frequency and Q-factor of a series LCR circuit
with $\text{L} = 3.0\, \text{H}$, $\text{C} = 27\, \mu\text{F}$, and $\text{R} = 7.4\, \Omega$. It is desired to improve the
sharpness of the resonance of the circuit by reducing its `full width at half maximum' by a factor of 2. Suggest a suitable way.\\
\\
\textbf{Solution: }
Given parameters are:

\begin{table}[h]
    \centering
    \resizebox{6 cm}{!}{
    \begin{tabular}{|c|c|}
    \hline
      \textbf{Vertex}   &  \textbf{Coordinate}\\
    \hline
       $P$  & $(2, 1)$\\
    \hline
        $Q$ & $(-2, 3)$\\
    \hline
        $R$ & $(4, 5)$\\
    \hline
\end{tabular}
}
    \vspace{6 pt}
    \caption{Given Parameters}
    \label{tab:my_label}
\end{table}

\begin{figure}[h]
 \centering
    \input{figs/fig1}
    \caption{LCR Circuit}
    \label{fig:enter-label}
\end{figure}

\textbf{Frequency Response of the Circuit}

Applying Kirchhoff's Voltage Law (KVL), we get:

\begin{equation}
V_R + V_L + V_C = V(t) \label{eq:KVL}
\end{equation}

where $V_R$, $V_L$ and $V_C$ are the voltages across R, L and C respectively and $V(t)$ is the time-varying voltage source.

\begin{table}[h]
    \centering
    \resizebox{6 cm}{!}{
    \begin{tabular}{|c|c|c|}
    \hline
     \textbf{Symbol} & \textbf{Reactance} &
     \textbf{Description}\\
    \hline
    R &  $R$ & Resistance\\
    \hline
    L &  $sL$ & Inductance\\
    \hline
    C & $\dfrac{1}{sC}$ & Capacitance\\[6pt]
    \hline
\end{tabular}

    }
    \vspace{6 pt}
    \caption{Impedances}
    \label{tab:my_label} 
\end{table}

Elements and their corresponding impedances are given in the table above where $s$ is a complex variable. The circuit can now be redrawn as:

\begin{figure}[h]
 \centering
    \input{figs/fig2}
    \caption{LCR Circuit}
    \label{fig:2}
\end{figure}

Using the impedances of R, L and C from TABLE II in equation \eqref{eq:KVL}, we get

\begin{equation}
    V(s) = R I(s) + sL I(s) + \dfrac{1}{sC} I(s)
\end{equation}

\begin{equation}
    \Rightarrow V(s) = I(s)\left(R + Ls + \dfrac{1}{sC}\right)
\end{equation}

\begin{equation}
    \Rightarrow I(s) = \dfrac{V(s)}{\left(R + Ls + \dfrac{1}{sC}\right)}
\end{equation}

At resonance, the circuit becomes purely resistive. The impedances of capacitor and inductor cancel out as follows:

\begin{equation}
    Ls + \dfrac{1}{sC} = 0
\end{equation}

\begin{equation}
    \Rightarrow s = j\dfrac{1}{\sqrt{LC}} \label{eq: 6}
\end{equation}

$s$ can be expressed in terms of angular resonance frequency as

\begin{equation}
    s = j\omega_0 \label{eq: 7}
\end{equation}

Comparing equations \eqref{eq: 6} and \eqref{eq: 7}, we get

\begin{equation}
    \omega_0 = \dfrac{1}{\sqrt{LC}}
\end{equation}\\
\\
\textbf{Quality Factor}

Quality Factor ($Q$) of an LCR circuit is defined as the ratio of voltage across inductor or capacitor to that across the resistor at resonance.

\begin{equation}
    Q = \left(\dfrac{V_L}{V_R}\right)_{\omega_0} = \dfrac{\lvert{sLI(s)}\rvert}{\lvert RI(s) \rvert}
\end{equation}

\begin{equation}
    \Rightarrow Q = \dfrac{1}{\sqrt{LC}}\dfrac{L}{R}
\end{equation}

\begin{equation}
    \Rightarrow Q = \dfrac{1}{R}\sqrt{\dfrac{L}{C}}
\end{equation}


\renewcommand{\thefigure}{\theenumi}
\renewcommand{\thetable}{\theenumi}

\end{document}
