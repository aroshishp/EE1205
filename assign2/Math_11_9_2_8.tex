% \iffalse
\let\negmedspace\undefined
\let\negthickspace\undefined
\documentclass[journal,12pt,twocolumn]{IEEEtran}
\usepackage{cite}
\usepackage{amsmath,amssymb,amsfonts,amsthm}
\usepackage{algorithmic}
\usepackage{graphicx}
\usepackage{textcomp}
\usepackage{xcolor}
\usepackage{txfonts}
\usepackage{listings}
\usepackage{enumitem}
\usepackage{mathtools}
\usepackage{gensymb}
\usepackage{comment}
\usepackage[breaklinks=true]{hyperref}
\usepackage{tkz-euclide} 
\usepackage{listings}
\usepackage{gvv}                                        
\def\inputGnumericTable{}                                
\usepackage[latin1]{inputenc}                            
\usepackage{color}                                       
\usepackage{array}                                       
\usepackage{longtable}                                   
\usepackage{calc}                              
\usepackage{tikz}
\usepackage{multirow}                                    
\usepackage{hhline}                                      
\usepackage{ifthen}                            
\usepackage{caption}
\usepackage{lscape}
\usepackage{amsmath}
\newtheorem{theorem}{Theorem}[section]
\newtheorem{problem}{Problem}
\newtheorem{proposition}{Proposition}[section]
\newtheorem{lemma}{Lemma}[section]
\newtheorem{corollary}[theorem]{Corollary}
\newtheorem{example}{Example}[section]
\newtheorem{definition}[problem]{Definition}
\newcommand{\BEQA}{\begin{eqnarray}}
\newcommand{\EEQA}{\end{eqnarray}}
\newcommand{\define}{\stackrel{\triangle}{=}}
\theoremstyle{remark}
\newtheorem{rem}{Remark}

\begin{document}

\bibliographystyle{IEEEtran}
\vspace{3cm}

\title{NCERT Math 11.9.2 Q8}
\author{EE23BTECH11009 - AROSHISH PRADHAN$^{*}$% <-this % stops a space
}
\maketitle
\newpage
\bigskip
\textbf{Question:} If the sum of $n$ terms of an AP is $(pn + qn^2)$, where $p$ and $q$ are constants, find the common difference.\\

\solution
\begin{table}[!h]
    \centering
    \begin{tabular}{|c|c|}
    \hline
      \textbf{Vertex}   &  \textbf{Coordinate}\\
    \hline
       $P$  & $(2, 1)$\\
    \hline
        $Q$ & $(-2, 3)$\\
    \hline
        $R$ & $(4, 5)$\\
    \hline
\end{tabular}

    \caption{Given Parameters}
    \label{tab:1}
\end{table}

Sum of $n$ terms, as a discrete signal:
\begin{equation}
    s(n) = (pn + qn^2)u(n)
\end{equation}

Taking the $Z$-Transform,
\begin{align}
    s(n) &\system{Z} S(z)\\
    \implies S(z) &= \sum_{n = - \infty}^{\infty}s(n)z^{-n}\\
    &= \sum_{n = - \infty}^{\infty}(pn + qn^2)u(n)z^{-n}\\
    &= p\sum_{n = -\infty}^{\infty}nu(n)z^{-n} + q\sum_{n = -\infty}^{\infty}n^2u(n)z^{-n}\\
      &= p\brak{\frac{z^{-1}}{(1-z^{-1})^2}} + q\brak{\frac{z^{-1}(1 + z^{-1})}{(1-z^{-1})^3}}
\end{align}

$\{\abs{z} > 1\}$

Now, 
\begin{align}
    s(n) &= x(n) \ast u(n)\\
    \implies S(z) &= X(z)U(z)\\
    \implies X(z) &= \frac{S(z)}{U(z)}\label{eq:9}
\end{align}

where,
\begin{align}
    U(z) &= \frac{1}{1 - z^{-1}}\label{eq:10}
\end{align}

$\{\abs{z} > 1\}$

Using \eqref{eq:10} in \eqref{eq:9},
\begin{align}
    X(z) &= p\brak{\frac{z^{-1}}{(1-z^{-1})}} + q\brak{\frac{z^{-1}(1 + z^{-1})}{(1-z^{-1})^2}}
\end{align}

Simplifying using partial fractions, we get:
\begin{align}
    X(z) &= (q-p) + \frac{p-3q}{1-z^{-1}} + \frac{2q}{(1-z^{-1})^2}\\
    &= (q - p) + \frac{(p-q)}{1-z^{-1}} + \frac{2qz^{-1}}{(1-z^{-1})^2}
\end{align}

Taking the inverse Z-Transform,
\begin{align}
    x(n) = (q-p)\delta(n) + (p-q)u(n) + 2qnu(n)\label{eq:15}
\end{align}
\\
To simplify, use $n=0$:
\begin{equation}
    s(0) = x(0) = 0
\end{equation}
\begin{align}
    &\implies (q-p)\delta(0) + (p-q)u(0) + 2qnu(0) = 0\\
    &\implies p = q
\end{align}

because $\delta(0) = 1$ and $u(0) = 0$

$\therefore$ rewriting \eqref{eq:15}:
\begin{equation}
    x(n) = 2qnu(n)
\end{equation}

Common difference is given by:
\begin{align}
    d &= x(n+1) - x(n)\\
    &= 2q(n+1)u(n+1) - 2qnu(n)\\
    &= 2q
\end{align}

\begin{figure}[!h]
    \centering
    \includegraphics[width = \columnwidth]{figs/x_plot.png}
    \caption{Plot of x(n) vs n}
    \label{fig:1}
\end{figure}
\end{document}
